
\documentclass[12pt]{article}
\usepackage[margin=1in]{geometry}
\geometry{letterpaper} 
\usepackage{amsmath}

\title{Statistical Connectomics Project Proposal: A Dynamic Stochastic Block Model for Analyzing Temporal Changes in the Human Connectome}
\author{David Lee}
\date{April 2, 2015}

\begin{document}
\maketitle
\paragraph{Opportunity}
The stochastic block model is an extremely important tool for analyzing the connectivity of different regions of the brain.  However, neural connections in the human brain are not static.  They may change over time due to aging or the onset of some neurodegenerative disease.  Therefore, a stochastic block model that could account for these temporal changes could be of great use to study blocks that may not be apparently when the brain is analyzed statically.

\paragraph{Challenge}
A classic stochastic block model is static and does not consider changes in edges over time.  In order to create a dynamic stochastic block model that factors in how edges form or disappear over time the model must be able incorporate these transitions.

\paragraph{Action}
This proposal plans the address the issues mentioned above by modifying the stochastic block model so that it considers edge formation and disappearance.  A graph with a set number of vertices that has its edges changes over time will be generated by a simulation on MATLAB and the dynamic stochastic block model will assign blocks at each time point.  Clustering at each time point will be examined and compared to the classic static stochastic block model.

\paragraph{Resolution}
An adjusted random index for the dynamic and the static stochastic block models will be compared at each discrete time point given from the MATLAB simulation.

\paragraph{Future Work}
The following proposed model will only use discrete time points for assessing connectivity.  A continuous time (or duration based) model could also be of interest as well.  Whether or not we decide to make our model assume conditional independence for all past graph also remains a question.  Although this proposal suggests using a static number of vertices and a dynamic number of edges, using a dynamic number of vertices may be more useful since the number of neurons in the brain can change over time.

\pagebreak
\paragraph{Statistical Decision Theoretic}
\begin{description}
\item[Sample space]
The sample space will be the set of graphs where each graph represents the verticies and edges at a certain time point.  This can be represented by a set of n by n adjacency matrices, $A \in \mathcal{A}_n $, where $A = \{0,1\}^{n \times n} \\ $

\item[Model]
We will modify the following stochastic block model given by the equation
\begin{equation*}
P=SBM_n^k(\rho,\beta)\textrm{ where }\rho\in\Delta_k\textrm{ and }\beta\in(0,1)^{k\times k}
\end{equation*}

\item[Action space]
$Z \in \mathcal{Z}_n $, where $Z = \{0,1\}^{n \times n} \\ $


\item[Decision rule class]
$ f: \mathcal{A}_n \to \mathcal{Z}_n$

\item[Loss function]
The loss function can be given by the adjusted rand index 
\begin{equation*}
l:G_n\times A\to R_+
\end{equation*}


\item[Risk function]
The risk function will be the expected value of the loss function
\begin{equation*}
R=P\times l
\end{equation*}
or alternatively 
\begin{equation*}
	R=E\{l\}
\end{equation*}
\end{description}
\end{document}	