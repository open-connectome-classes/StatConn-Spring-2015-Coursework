\documentclass[12pt]{article}
\usepackage[left=0.8in,right=0.8in,top=0.7in, bottom=0in]{geometry}
\geometry{letterpaper}     
\usepackage[hyphens]{url}
\usepackage{fancyhdr}
\pagestyle{fancy}
\usepackage[parfill]{parskip}    % Activate to begin paragraphs with an empty line rather than an indent
\usepackage{graphicx}
\usepackage{amssymb}
\usepackage{amsmath}
\usepackage{mathrsfs}
\usepackage{epstopdf}
\usepackage{url}
\DeclareGraphicsRule{.tif}{png}{.png}{`convert #1 `dirname #1`/`basename #1 .tif`.png}

\title{Statistical Connectomics: A posed statistical decision theoretic for brain graph clustering}
\author{Greg Kiar}
\begin{document}
\pagenumbering{gobble}
\setlength{\voffset}{-0.5in}
\setlength{\headsep}{5pt}
\maketitle
\vspace{-1cm}
\section*{Overview}
A relevant task in brain graph classification is the clustering and recognition of similar cliques within the graph. Here, we pose a statistical decision theoretic to aid in the development and evaluation of such clustering techniques. As we know, a statistical decision theoretic must define the following: a sample space, model, action space, decision rule class, loss function, and risk function. Shown below are these elements of the theoretic designed.

\section*{Posed Statistical Decision Theoretic} 
\textbf{Sample Space} $\mathcal{G}_n = (V, E, Y)$ \\
The sample space for clustering is the space of graphs with fixed vertices, edges between them, and a label indicating one of two categories.

\textbf{Model} $P = SBM^k_n (\rho, \beta)$, where, for $k=2$: $\rho \in \Delta_2$, $\beta \in (0,1)^{2x2}$\\
The model used to simulate data within this space is the stochastic block model with two distributions.

\textbf{Action Space} $A = \{ y \in \{0, 1\}^n \} $
The action space consists of cluster assignments made by the decision rule class.

\textbf{Decision Rule Class} The clustering algorithm of choice can be stated and implemented here.

\textbf{Loss Function} $l: \mathcal{G}_n \times A \to R_+$, $l = \sum\limits_{i=1}^n \Theta(\hat{y}_i = y_i)$ \\
Loss in this statistic can be measured by observing incorrect assignment of class. The adjusted rand index (ARI) is a good measure to do this, as it allows you to compute a normallized to $\{0,1\}$ loss dependent on the number of correcty and incorectly assigned labels.

\textbf{Risk Function} $R = P \times l$, e.g. $R = E_P(l)$\\
The risk function can, in the simple case, be the expected loss of the function. 

\end{document}  