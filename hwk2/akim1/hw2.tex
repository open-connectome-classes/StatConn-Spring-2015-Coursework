\documentclass{article}

\usepackage{amsmath}
\usepackage{mathrsfs}

\title{Statistical Decision Framework}
\author{akim1}
\date{February 17, 2015}

\begin{document}
\maketitle

As described in lecture, there are five components to the statistical decision
theoretical problem. One example of such is described below for the problem of
clustering nodes in a graph.

\begin{description}
\item[Sample space]
The sample space describes all possible arrangements of edges, nodes, and
labels.
\begin{equation*}
\mathscr{G}=(V,E,Y)
\end{equation*}

\item[Model]
The model is given by the two stochastic blocks and can be described as follows:
\begin{equation*}
P=\mathit{SBM}_n^2(\rho,\beta)\textrm{ where }\rho\in\Delta_2\textrm{, }\beta\in(0,1)^{2\times2}
\end{equation*}

\item[Action space]
The action space is the assignment given by the clustering algorithm.
\begin{equation*}
A=\{y\in\{0,1\}^n\}
\end{equation*}

\item[Decision rule class]
The decision rule class is given by k-means square clustering.

\item[Loss function]
The loss function is given by the following using the adjusted rand index:
\begin{equation*}
l=\sum_{i=1}^n\Theta(\hat{y}_i=y_i)
\end{equation*}

\item[Risk function]
The risk function is simply thee expected value of the loss function:
\begin{equation*}
	R=E\{l\}
\end{equation*}
\end{description}
\end{document}
